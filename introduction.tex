\section{Introduction} \label{sec:intro}
Cyber-Physical Systems(CPS) and Open-Source Intelligence(OSINT) are explained in \cref{sec:definitions}. While a lot of the Cyber-Physical Systems connected to the internet are well maintained and secure, many other are vulnerable. This is most often because the CPSs are not set up properly or run outdated software. Other than this, it is a "fact" of statistics that if a quantity CPSs are connected to the internet, a portion of these will be insecure. To explore the extent of this challenge, the first step is to identify how many CPSs can be reached. Constraints are set to limit the focus to \textbf{discoverable Cyber Physical Systems in the maritime and offshore industries.}
Open-source intelligence(OSINT) is utilized to find these CPS. The OSINT search engine \href{https://shodan.io}{\color{blue}{Shodan}} is a popular search engine for enumerating Internet-connected systems, and will be used as the main reconnaissance tool of this project.

\subsection{Collaboration}\label{sec:collaboration}
This project is a collaboration between the main supervisor Mary Ann Lundteigen and PhD candidate Bálint Zoltán Téglásy, both representing NTNU; and the Global Service Line leader, Cybersecurity, at DNV-GL Trondheim,  Mate J Csorba. The student carrying out the project is Sigurd Hellesvik.

\subsubsection{DNV-GL}\label{sec:dnvgl}
DNV-GL are a independent expert in risk management and quality assurance. The abbreviation is for "Det Norske Veritas" and "Germanischer Lloyd". A big portion of their focus is on the maritime and offshore business. The DNV-GL has a Cyber Security team that helps customers assess and manage risks related to cyber security.~\cite{DNVGL_cybersec}  The International Maritime Organization (IMO) "encourages administrations to ensure that cyber risks are appropriately addressed in existing safety management systems (as defined in the ISM Code) no later than [...] 1 January 2021".~\cite{IMO_2021} Because of this, many maritime corporations will have to pay more attention to cyber security; making maritime cyber security specialists like DNV-GL Cyber Security team all the more relevant for the business. 


\subsubsection{NTNU}\label{sec:ntnu}
The Norwegian University of Technology and Science, with the Norwegian abbreviation NTNU, is the largest university in Norway, and has 75 percent of the technology master candidates in the country. This projects mainly takes place on the Trondheim Campus.


\subsection{Project goal}\label{sec:goal}
\textbf{The goal of this project is to map Cyber-Physical Systems(CPS) in the European maritime and offshore industry that is reachable through the internet, and by doing so get an overview of the exposure of these cyber-physical systems.}


\subsection{Definition of important terms} \label{sec:definitions}

\subsubsection{Cyber-Physical Systems(CPS)}\label{sec:cps}
Cyber-Physical Systems(CPS) can be defined as "engineered systems that integrate information technologies, real‐time control subsystems, physical components, and human operators to influence physical processes by means of cooperative and (semi)automated control functions."~\cite{guzman_wied_kozine_lundteigen_2019}
CPS is a broad term, and examples can be everything automated control systems at offshore platforms to the computer system in an electric car. As more and more of the CPSs are connected to the internet, cyber threats becomes a bigger challenge. 

\subsubsection{Open-Source Intelligence(OSINT)}\label{sec:osint}
Open-Source Intelligence(OSINT) is defined by the American Office of The Director of National Intelligence in their Intelligence Community Directive as: "[Intelligence] produced from publicly available information that is collected, exploited, and disseminated in a timely manner to an appropriate audience for the purpose of addressing a specific intelligence requirement."\cite{directive_301} This directive also defines Open Source Information as "Publicly available information that anyone can lawfully obtain by request, purchase, or observation.". Open-Source Intelligence (OSINT) is therefore gathering of intelligence that is lawfully available to the public. 


\subsection{Project definition}



