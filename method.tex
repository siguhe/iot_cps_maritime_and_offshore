\section{Selective identification of IP devices} \label{sec:method}
As described in \cref{sec:research_approach}, methods are suggested to find devices as per the project goal. 
This chapter describes these suggested methods, based on the theory from \cref{sec:theory}.

Several methods to filter devices based on what industries they belong to, are suggested in \cref{sec:identify_industry}. These methods are not exclusive, and multiple methods may for example be combined to double-check the validity of the results. In \cref{sec:identify_cps}, additional methods are suggested on how to identify if devices are part of a Cyber-Physical System.

\subsection{Identifying devices from specific industries} \label{sec:identify_industry}
\subsubsection{Banner similarities} \label{sec:banner_method}
When connecting a device to the internet, it needs to be configured. The configurer needs competence and time to do this properly. To make such devices more accessible and efficient, ready-made solutions are most often used. Due to this, the devices mainly have the same default configuration, and therefore identical banners. This can be used to identify devices by using the following method.
\begin{enumerate}
    \item Choose a device type that fulfills the constraints and may be connected to the internet.
    \item Find the IP address of one instance of such a device, and find its Shodan entry.
    \item Get its banner.
    \item Use unique information from the banner to find all devices with similar banners.
\end{enumerate}
The most difficult of these steps is to find the IP address. The easiest way to do this is to be in possession of, and have ownership to, the device, and find its IP address through administrative privileges. 
If this is not an option, the device could be found by guessing what information is in its banner, for example the device name or open ports.
Unfortunately, this project does not have access to devices that fit the goals described in \cref{sec:objectives}.

\subsubsection{Internet Service Provider and IP ranges} \label{sec:isp_method}
Internet Service Providers(ISP) are the organizations that connect people and companies to the internet. ISPs charge money to deliver internet connectivity, and most people have a subscription. Telenor is for example a Norwegian ISP. While a lot of ISPs are generalized and deliver the internet to both private and organizational customers, others can be more specialized. 

Regional Internet Registries (RIR) sell IP addresses to organizations, for them to distribute further. For example, an Internet Service Provider(ISP) will buy a CIDR range, then give IPv4 addresses to customers while they use their subscribed internet connections. An overview of the distribution of IPv4 addresses to organizations is illustrated in \cref{fig:ipv4_map}. Shodan can use the filter "isp" to search for ISPs, "org" to search for organizations, and "net" to search for devices in a CIDR range. 

\begin{figure} [H]
    \centering
    \includegraphics[scale=4]{Figurer/map_of_the_internet.jpg}
    \caption{A map of the IPv4 space, from the XKCD webcomic. \cite{xkcd} }
    \label{fig:ipv4_map}
\end{figure}

While it often is an uncomplicated procedure to find the name of an organization or ISP, a tool is needed to find CIDR ranges. One or more CIDR ranges owned by an organization and with a single defined routing policy, is called an Autonomous System(AS). A typical example of an organization that owns an AS, is an Internet Service Provider(ISP). \cite{AS_def} 
The ASs are registered with unique identification numbers. Online tools, like \href{https://hackertarget.com/as-ip-lookup/}{\textbf{Hackertarget AS-IP lookup}} \cite{asip_lookup}, make it possible to find the parent AS of an IP address. Then the rest of the IP addresses belonging to the same AS can be found.


\subsubsection{Reverse IP geolocation} \label{sec:geo_method}
When Regional Internet Registries (RIR) give IP addresses to organizations, they also register contact information on the IP addresses. Part of this information is a location, specifying a country and a city. Some data providers claim that the country accuracy is between 95\% and 99\%, while the city accuracy is between 50\% and 75\%.\cite{geolocation_acc} With accuracy that poor, it is not possible to use IP geolocation to decisively determine the location an IP address. The information returned by one of the RIRs can be seen in \cref{fig:RIPE_NCC}.
Shodan has a filter for the location. This filter can also query by latitude and longitude. However, Shodan does not disclose how they gather this information. 
Because of these uncertainties, IP geolocation can not be conclusive for identifying devices. However, by reversing the technique, it would be possible to narrow down the selection of IP addresses by first defining a geographical area. Shodan can use the filter "geo:latitude,longitude,radius" to filter addresses by location.  

\begin{figure} [H]
    \centering
    \includegraphics[scale=0.5]{Figurer/ripe.png}
    \caption{The IP address lookup of RIPE NCC \cite{ripe_whois}}
    \label{fig:RIPE_NCC}
\end{figure}

\subsubsection{Latency and traceroute} \label{sec:latency_method}
The simplest form of reaching another device on the internet is called a "ping". This sends a echo request to an IP address, to make it send a response back. It is mainly used to check if a device is reachable. To prevent a bug where packets travel in an infinite loop instead of reaching the recipient, packets have a Time To Live (TTL) counter. The packet increments a variable every time a it travels through a switch or router. When this variable reaches the TTL counter , the packet is stopped, and a message is returned to the sender, informing it of the loss.
The "traceroute" command takes advantage of the fact that the Time To Live counter can be set by the sender. It will start the TTL counter at 1 and will send a message towards a predefined target IP address. This message will then travel 1 step towards its target before it is returned. Then traceroute increments this counter and repeats the send. This way, all routers that the packet travels through on its way, will be indexed. In addition, traceroute will time how long the packet uses before it is returned, called "Round Trip Time"(RTT).
The Round Trip Time can give an idea of how the signal travels. Faster RTT can the result of shorter distance or better infrastructure. Vice versa, slow RTT may indicate longer distance or worse infrastructure.
The traceroute is visualized in \cref{fig:latency}. Here the signal will travel fast from Source to both Router 1 and Router 2. There will then be a jump in latency when the signal crosses the Atlantic ocean on its way to router 3. The longest delay will be where the signal travels to the Destination from Router 3 via a satellite.

\begin{figure} [H]
    \centering
    \includegraphics[scale=0.3]{Figurer/latency.png}
    \caption{Visualization of internet latency. Map from \href{http://getdrawings.com/earth-cartoon-drawing}{http://getdrawings.com/earth-cartoon-drawing}}
    \label{fig:latency}
\end{figure}

\subsection{Identifying devices from Cyber-Physical Systems} \label{sec:identify_cps}
As mentioned in \cref{sec:cps}, Cyber-Physical Systems(CPS) are systems containing anything from real-time controllers to human operators. Not all CPS have components that are connected to the internet. When a CPS is connected to the internet, only some of its components are actually connected. In the offshore industry, CPSs are typically Industrial Control Systems(ICS). In the maritime industry, CPSs are typically navigation systems. As these industries often overlap, their CPS use-cases often overlap as well. Methods for identifying two subsections of CPS, ICS and navigation systems will be proposed below.
These methods can be used separately, but in this project they are intended for use on a subset of devices that are filtered from the methods suggested in \cref{sec:identify_industry}.

\subsubsection{Industrial Control Systems}
Industrial Control Systems(ICS) are in most cases easy to identify. A lot of previous research already exists on how to use Shodan to identify ICS. For example: \textit{Exploring Shodan From the Perspective of Industrial Control Systems}\cite{bodenheim_butts_dunlap_mullins_2014} and \textit{Evaluation of the ability of the Shodan search engine to identify Internet-facing industrial control devices} \cite{ICS_shodan_article}.  In addition, Shodan has a list of popular ICS and search terms used to find the systems.\cite{shodan_ics} The different ICSs use standard ports for communication through the internet, and Shodan can use the "port" filter to find them. ICSs often have similar banners, and the same method as in \cref{sec:banner_method} would work on ICSs.

\subsubsection{Navigation systems}
In the maritime industry, Electronic Chart Display and Information System(ECDIS) is often used for navigating ships. While a lot of them have downloaded maps, many also use online maps, and therefore has a connection to the internet. For identifying these systems, the methods proposed in \cref{sec:banner_method} and \cref{sec:isp_method} would be the most effective.

\newpage
