\section{Further work} \label{sec:further_work}
When the approaches for finding devices connected to the internet, described in \cref{sec:method} was implemented in \cref{sec:results}, the implementations were manual. This was intended to show an example of the approach. It should be possible to implement these approaches as some kind of crawlers, or parsers for datasets.

Information gathering from the internet is a popular business, and datasets about a lot of different things should be available. For example, as mentioned in \cref{sec:latency_results}, a dataset for traceroute data from all available IP addresses would be useful. 

As mentioned in \cref{sec:limits}, Shodan is not the only alternative for use as crawler. It would be interesting to see if \href{https://censys.io/}{\color{blue}{Censys}}\cite{censys} and \href{www.zoomeye.org}{\color{blue}{ZoomEye}}\cite{zoomeye} would provide different results
