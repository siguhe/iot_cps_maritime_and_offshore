\section{Further work} \label{sec:further_work}
When the methods for finding devices connected to the internet, described in \cref{sec:method} were implemented in \cref{sec:results}, the implementations were manual. The intention of this choice was to show an example of the method. It ought to be possible to implement these methods as some kind of parsers for datasets or crawlers.

Gathering information from the internet is a popular business, and datasets about a lot of different topics are most likely available. For example, as mentioned in \cref{sec:latency_results}, a dataset for traceroute data from all available IP addresses would be useful. 

As mentioned in \cref{sec:limits}, Shodan is not the only alternative for use as crawler. It would be interesting to see if \href{https://censys.io/}{\color{blue}{Censys}}\cite{censys} and \href{www.zoomeye.org}{\color{blue}{ZoomEye}}\cite{zoomeye} would provide different results.
