\section{Developement}
This report is under developement. Each week, the new parts will be added after \cref{sec:new}. The other parts of the report can be changed at any time, but in such a case, they will be moved back to \cref{sec:new}.


\section{Introduction} \label{sec:intro}
While a lot of the Cyber-Physical Systems(CPS) connected to the internet are well maintained and secure, many other are vulnerable. This is most often because the CPSs are not set up properly or run outdated software. Other than this, it is a "fact" of statistics that if a quantity CPSs are connected to the internet, a portion of these will be insecure. To explore the extent of this challenge, the first step is to identiy how many CPSs can be reached. The search engine \href{https://shodan.io}{\color{blue}{Shodan}} is a popular search engine for enumerating Internet-connected systems. Because of the sheer number of  publicly accessible devices, not all can be mapped here, and constraints have to be set. These constraints decide a focus on CPS belonging to the European maritime and offshore industries.

\subsection{Collaboration}\label{sec:collaboration}
This project is a collaboration between the main supervisor Mary Ann Lundteigen and PhD candidate Bálint Zoltán Téglásy, both representing NTNU; and the Global Service Line Leade, Cybersecurity, at DNV-GL Trondheim,  Mate J Csorba. The student carrying out the project is Sigurd Hellesvik.

\subsubsection{DNV-GL}\label{sec:dnvgl}
DNV-GL are a independent expert in risk management and quality assurance. The abbreviation is for "Det Norske Veritas" and "Germanischer Lloyd". A big portion of their focus is on the maritime and offshore business. The DNV-GL has a Cyber Security team that helps customers assess and manage risks related to cyber security.~\cite{DNVGL_cybersec}  The International Maritime Organisation (IMO) "encourages administrations to ensure that cyber risks are appropriately addressed in existing safety management systems (as defined in the ISM Code) no later than [...] 1 January 2021".~\cite{IMO_2021} Because of this, many maritime cooperations will have to pay more attention to cyber security; making maritime cyber security specialists like DNV-GL Cyber Security team all the more revelant for the business. 

\subsection{Project goal}\label{sec:goal}
\textbf{The goal of this project is to map Cyber-Physical Systems(CPS) in the European maritime and offshore industry that is reachable through the internet, and by doing so get an overview of the exposure of these cyber-physical systems.}



