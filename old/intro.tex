\section{Development}
This report is under development. Each week, the new parts will be added after \cref{sec:new}. The other parts of the report can be changed at any time, but in such a case, they will be moved back to \cref{sec:new}.

\subsection{Collaboration}\label{sec:collaboration}
This project is a collaboration between the main supervisor Mary Ann Lundteigen and PhD candidate Bálint Zoltán Téglásy, both representing NTNU; and the Global Service Line leader, Cybersecurity, at DNV-GL Trondheim,  Mate J Csorba. The student carrying out the project is Sigurd Hellesvik.

\subsubsection{DNV-GL}\label{sec:dnvgl}
DNV-GL are a independent expert in risk management and quality assurance. The abbreviation is for "Det Norske Veritas" and "Germanischer Lloyd". A big portion of their focus is on the maritime and offshore business. The DNV-GL has a Cyber Security team that helps customers assess and manage risks related to cyber security.~\cite{DNVGL_cybersec}  The International Maritime Organization (IMO) "encourages administrations to ensure that cyber risks are appropriately addressed in existing safety management systems (as defined in the ISM Code) no later than [...] 1 January 2021".~\cite{IMO_2021} Because of this, many maritime corporations will have to pay more attention to cyber security; making maritime cyber security specialists like DNV-GL Cyber Security team all the more relevant for the business. 

\subsection{Project goal}\label{sec:goal}
\textbf{The goal of this project is to map Cyber-Physical Systems(CPS) in the European maritime and offshore industry that is reachable through the internet, and by doing so get an overview of the exposure of these cyber-physical systems.}



