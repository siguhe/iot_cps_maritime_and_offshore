\section{Literature} \label{sec:literature}
\subsection{Articles} \label{sec:articles}
\subsubsection{Cybersecurity and Safety Co-Engineering of Cyberphysical systems -A Comprehensive Survey}
This article look into how recent methods to implement cybersecurity and Safety in CPS. Safety and cybersecurity are often interconnected, and affect the same systems. While they often are complemnentary, other times safety and cybersecurity are in conflict. The aricle in question use a good example, where a door will open on power loss to ensure the safety of workers. However, cybersecurity want the same door to close on power loss, to stop intruders. The article look at a sample of studies on the subject from the last 20 years, and compares them. The conclusion is to see what topics need more research. How can this be relevant for the report?

\subsubsection{Shodan Vizualized}
The article \textit{Shodan Visualized} gives a map of SCADA devices connected to the internet.\cite{ercolani_patton_chen_2016} This is good to prove the point that a lot of CPSs are connected. For myself: Look into difference between CPS and SCADA. 

\subsubsection{Evaluation of the ability of the Shodan search engine to identify Internet-facing industrial control devices}
\textit{Evaluation of the ability of the Shodan search engine to identify Internet-facing industrial control devices} gives a good overview of how Shodan works, with an explanation of the Shodan process for indexing webpages.\cite{bodenheim_butts_dunlap_mullins_2014} 
In addition, it shows how Shodan can be used to search for Industrial Control Devices(ICS). A lot of those ICS will be used in the offshore industry, and some in the maritime industry. 
Lastly, the report shows that obfuscating devices make them harder to find using Shodan.

\subsection{Standards} \label{sec:standards}
\subsubsection{IEC 62443} \label{sec:IEC62443}
The IEC 62443 series is a series of technical reports and standards that defines procedures for how to implement secure Industrial Automation and Control Systems (IACS), focusing on the cybersecurity of these systems. This series covers multiple industries.

\subsubsection{DNVGL-RP-G108} \label{sec:G108}
Due to the scope of IEC 62443 being general, it can be time consuming to implement it for different standards.
For the Oil and Gas sector, the DNVGL-RP-G108 is a recommended practice for implementing IEC 62443. While the implementation of cybersecurity functions are not relevant to for mapping CPS, it can be useful to know more about the different systems that can be exposed to the internet. In that way, it can be discussed which procedures of the DNVGL-RP-G108 are used to secure IACS against eventual threats coming trough internet connections.
