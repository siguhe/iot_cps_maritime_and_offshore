\section*{Preface}
\addcontentsline{toc}{section}{Preface}

\section*{Acknowledgments} \label{sec:ack}
\addcontentsline{toc}{section}{Acknowledgments}
This project is a collaboration between the main supervisor Mary Ann Lundteigen and PhD candidate Bálint Zoltán Téglásy, both representing NTNU; and the Global Service Line leader, Cybersecurity, at DNV-GL Trondheim,  Mate J Csorba. The student carrying out the project is Sigurd Hellesvik.

\subsection*{DNV-GL}\label{sec:dnvgl}
\addcontentsline{toc}{section}{DNG-GL}
DNV-GL are a independent expert in risk management and quality assurance. The abbreviation is for "Det Norske Veritas" and "Germanischer Lloyd". A big portion of their focus is on the maritime and offshore business. The DNV-GL has a Cyber Security team that helps customers assess and manage risks related to cyber security.~\cite{DNVGL_cybersec}  The International Maritime Organization (IMO) "encourages administrations to ensure that cyber risks are appropriately addressed in existing safety management systems (as defined in the ISM Code) no later than [...] 1 January 2021".~\cite{IMO_2021} Because of this, many maritime corporations will have to pay more attention to cyber security; making maritime cyber security specialists like DNV-GL Cyber Security team all the more relevant for the business. 

\subsection*{NTNU}\label{sec:ntnu}
\addcontentsline{toc}{section}{NTNU}
The Norwegian University of Technology and Science, with the Norwegian abbreviation NTNU, is the largest university in Norway, and has 75 percent of the technology master candidates in the country. This projects mainly takes place on the Trondheim Campus.



\section*{Summary}
\addcontentsline{toc}{section}{Summary}
